\section{Conclusiones}

Creemos a partir de los resultados que el modelado de la fuente $S1$ partiendo de los mensajes Who-has distinguiendo origen es un buen punto de partida para descubrir topologías en redes inalámbricas. Si bien el modelo funcionó bien para redes de distintos tamaños, en el caso que la red analizada sea muy compleja y/o los resultados poco claros puede complementarse con otras técnicas para facilitar la comprensión correcta de la red.\\

También vemos que, en base a los resultados de la red cableada, este modelo no necesariamente sea el indicado para analizar redes de este tipo. Como mencionamos en el desarrollo para esa red analizada obtuvimos mejores resultados con el modelo alternativo $S2$, queda para trabajos futuros indagar si la hipótesis es correcta.\\

En cuanto a la relación de la entropía y la distinción de nodos, creemos que es una condición necesaria que la entropía sea relativamente baja comparada con la entropía teórica máxima, suponemos que con que sea menor que la mitad alcanza. Si esto no fuese así la información que proporcionan los símbolos serían similares entre si y por lo tanto dificultaría la tarea de buscar nodos distinguidos.\\

Luego, mientras más símbolos haya, mayor va a ser la entropía máxima teórica. La relación entre la entropía real y la cantidad de nodos viene de la decisión que tomamos de utilizarla como punto de corte para separar entre nodos distinguidos y no distinguidos.

\addtolength{\textheight}{-12cm}   % This command serves to balance the column lengths
% on the last page of the document manually. It shortens
% the textheight of the last page by a suitable amount.
% This command does not take effect until the next page
% so it should come on the page before the last. Make
% sure that you do not shorten the textheight too much.



%%%%%%%%%%%%%%%%%%%%%%%%%%%%%%%%%%%%%%%%%%%%%%%%%%%%%%%%%%%%%%%%%%%%%%%%%%%%%%%%