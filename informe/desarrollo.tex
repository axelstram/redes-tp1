\section{Desarrollo}

%Para realizar este TP utilizamos como herramientas Wireshark[1], Scapy[2], y Python. 
Para el presente trabajo práctico se desarrolló un programa en Python utilizando la libreria Scapy que captura los paquetes que escucha la interfaz definida y filtra los mismos quedandose solamente con aquellos que son paquetes del protocolo ARP, el mismo programa acepta como parámetros la ruta de un archivo en formato ''.pcap'' que puede ser generado por medio de capturas anteriores o utilizando software alternativo como Wireshark.

Tenemos dos variantes del programa, uno para cada fuente explicadas más adelante. Con los datos obtenidos por el programa se calculan la entropía y la cantidad de información de cada símbolo en cada fuente. En realidad el programa que modela la fuente ''s1'' es genérico y permite modelar diversas fuentes de información en base a los parámetros seleccionados en tiempo de ejecución.

Adicionalmente se utilizaron programas auxiliares para generar los gráficos, se crearon archivos dot por medio de scripts en python y luego se graficaron mediante GraphViz.

Se capturaron redes de distintos tamaños utilizando distintas tecnologías a nivel enlance: Switched Ethernet y Wireless LAN


\subsection{Fuente S}
Esta fuente binaria está compuesta por los símbolos ${s_{Broadcast}, s_{Unicast}}$ pertenecientes al protocolo ARP. Como sus nombres lo indican, el símbolo $s_{Broadcast}$ es un paquete que está destinado a toda la red (mensaje ARP "Who-has"), mientras que el símbolo $s_{Unicast}$ (mensaje ARP "Is-at") corresponde al paquete ARP que se envía como respuesta al mensaje "Who-has".

\subsection{Fuente S1}
Para S1 teníamos varias opciones dentro de los paquetes ARP. Podíamos ver los paquetes Who-Has o Is-At, así como también podíamos centrarnos en source o destino. Es decir, cuatro combinaciones. Para decidirnos por una de ellas lo que hicimos fue experimentar con todas y analizar los resultados. Nos terminamos quedando con las IP \textit{source} de los mensajes \textit{Who-has}, ya que ésta fuente fue la que mejor modelaba en general las redes con las que experimentamos. De todas maneras queremos notar que ninguna de las fuentes es perfecta, y que en algunas redes funcionan mejor otras fuentes (por ejemplo quedandonos con las IPs destino en vez de source), pero por lo dicho anteriormente terminamos eligiendo los paquetes source. En los casos en que funcionaron mejor fuentes alternativas Incluimos un análisis de la red con la fuente elegida, y otro con la fuente alternativa aventurando alguna conclusión.

Como se mencionó anteriormente, al ser el software de naturaleza genérica, para modelar la fuente escogida como s1, las opciones que hay que pasarle al programa son \textit{Who-has} y \textit{Source}.


\subsection{Implementación}
Para llevar a cabo los experimentos implementamos una herramienta en Python utilizando la librería \textit{scapy}[1], y capturamos los paquetes con \textit{Wireshark}[2].

%%explicacion de como funciona el programa.
Nuestro programa comienza preguntando si se desea capturar paquetes Who-has o Is-at, y a su vez si se quiere filtrar por origen o destino. Luego, crea 2 diccionarios: \textit{nodos}, que es un diccionario de la forma ${host: cant de apariciones}$, y \textit{connections}, que es de la forma ${src: [dst] }$ ó ${dst: [src] }$, según qué modo se elija. Éste último diccionario guarda, para cada key, una lista de todos los nodos con los que se conecta. Esto nos sirve luego para poder visualizar la red como un grafo.

Luego calculamos la entropía y la información que aporta cada nodo. Esto nos sirve para poder dividir los nodos en 2 categorías: \textbf{distinguidos} y \textbf{no distinguidos}. Los nodos distinguidos los definimos como aquellos cuya información está por debajo de la entropía. Lo que esperamos conseguir con esto es que entre los nodos distinguidos esté el default gateway. Probamos también con otros criterios de corte para considerar un nodo distinguido o no: por ejemplo, en vez de la entropía, tomar el logaritmo de la entropía, la raíz cuadrada de la entropía, dividir la entropía por una constante, etc. Por motivos de espacio no colocamos los resultados para cada uno de estos casos, pero no observamos ninguna mejoría notable con respecto a simplemente tomar la entropía. En algunos casos funcionaba mejor tomar simplemente la entropía, en otros el logaritmo de la entropía, pero en general funcionó mejor lo primero, y por lo tanto nos terminamos quedando con eso.

Calculado esto, procedemos a realizar distintos gráficos: el primero consiste en un gráfico de la información de cada nodo junto con la entropía máxima y real de la red. Nos quedamos con los 8 nodos con menor información, lo cuál resultó ser una buena heurística para observar tanto los nodos distinguidos como algunos nodos no distinguidos. Para éste gráfico utilizamos la librería \textit{matplotlib}. Otro gráfico que realizamos es un grafo con todos los nodos de la red y sus conexiones. Este lo realizamos con la librería \textit{networkx}.

Las redes que utilizamos para nuestros análisis son: una red corporativa grande, dos redes corporativas pequeñas, y una red hogareña.

A lo largo de las sucesivas experimentaciones notamos que las redes grandes y complejas generaban graficos e información muy difíciles de analizar sin acudir a algún tipo de resumen por lo cual, se agregó una opción para resumir nodos no distinguidos en subredes. Desde el punto de vista de código, se acudió a una simplificación primaria que fue suponer todas las subredes como redes /24 (a pesar de que somos conscientes que existen redes más chicas en la realidad), luego se utilizó la libreria \textit{netaddr} para definir dichas subredes /24 y testear pertenencia de un host a dichas redes y la funcion \textit{cidr\_merge} para agrupar subredes /24 contiguas en redes más grandes. La selección entre sumarizar los nodos no distinguidos o no hacerlos, queda a criterio del usuario a quién se le pide que seleccione una opción en tiempo de ejecución.

Luego de implementada esta \textit{sumarización} de redes, notamos que los gráficos aún no eran lo suficientemente representativos de la realidad por lo que acudimos a modificar la generación de los mismos con el objetivo de que visualmente aporten más información. Lo que se implementó fue, distinguir mediante colores aquellos nodos distinguidos (verde) de aquellos que no lo son (azul), y a dibujar cada nodo en escala dependiendo de la cantidad de mensajes que haya generado según el modelo. Cuando los gráficos se generan a partir de los nodos sin sumarizar, el tamaño de cada nodo distinguido es inversamente proporcional a la cantidad de información que aporta; en el caso de los nodos no distinguidos, depende de la cantidad de ejes que posee el nodo (esta cantidad es la suma entre los mensajes que lo tienen como destino y aquellos que los tiene como origen). En el caso de las redes sumarizadas, los nodos distinguidos se comportan de la misma manera, y lo no distinguidos dependen de la cantidad de equipos que estan dentro de la red, esto se hizo para que el tamaño de la \textit{nube} que representa la red sea representativo de la cantidad de equipos activos que la componen.

\subsection{Utilización de la herramienta}
Para correr el programa, se debe ejecutar el siguiente comando: python s1.py \textit{paquete}, donde \textit{paquete} es una captura en formato .pcap. De no especificarse éste parámetro, se realiza una captura en vivo de la red.

En ambos casos el usuario debe responder a los prompts de la aplicación para indicar cuál es el modelo de fuente que desea utilizar además de las opciones de visualización.
