\section{Desarrollo}

%Para realizar este TP utilizamos como herramientas Wireshark[1], Scapy[2], y Python. 
Para el presente trabajo práctico se desarrolló un programa en Python utilizando la libreria Scapy que captura los paquetes que escucha la interfaz definida y filtra los mismos quedandose solamente con aquellos que son paquetes del protocolo ARP, el mismo programa acepta como parámetros la ruta de un archivo en formato ''.pcap'' que puede ser generado por medio de capturas anteriores o utilizando software alternativo como Wireshark.

Tenemos dos variantes del programa, uno para cada fuente explicadas más adelante. Con los datos obtenidos por el programa se calculan la entropía y la cantidad de información de cada símbolo en cada fuente.

Adicionalmente se utilizaron programas auxiliares para generar los gráficos, se crearon archivos dot por medio de scripts en python y luego se graficaron mediante GraphViz.

Se capturaron redes de distintos tamaños utilizando distintas tecnologías a nivel enlance: Switched Ethernet y Wireless LAN


\subsection{Fuente S}
Esta fuente binaria está compuesta por los símbolos ${s_{Broadcast}, s_{Unicast}}$. 

\subsection{Fuente S1}
Para S1 teníamos varias opciones dentro de los paquetes ARP. Podíamos ver los paquetes Who-Has o Is-At, así como también podíamos centrarnos en source o destino. Es decir, cuatro combinaciones. Para decidirnos por una de ellas lo que hicimos fue experimentar con todas y analizar los resultados. 


[EXPLICACIÓN DE CON CUÁL NOS QUEDAMOS Y POR QUÉ]

[SIMBOLOS DE LA FUENTE]