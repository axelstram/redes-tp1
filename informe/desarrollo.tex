\section{Desarrollo}

%Para realizar este TP utilizamos como herramientas Wireshark[1], Scapy[2], y Python. 
Para el presente trabajo práctico se desarrolló un programa en Python utilizando la libreria Scapy que captura los paquetes que escucha la interfaz definida y filtra los mismos quedandose solamente con aquellos que son paquetes del protocolo ARP, el mismo programa acepta como parámetros la ruta de un archivo en formato ''.pcap'' que puede ser generado por medio de capturas anteriores o utilizando software alternativo como Wireshark.

Distintas variantes del programa permiten clasificar los paquetes capturados de acuerdo a la distinción entre paquetes con direccionamiento Broadcast o direccionamiento Unicast así como distinguiendo por direcciones IP de origen para modelar las fuentes $S$ y $S1$ respectivamente. Con estos datos se calculan la entropía y la cantidad de información de cada símbolo.

Adicionalmente se utilizaron programas auxiliares para generar los gráficos, se crearon archivos dot por medio de scripts en python y luego se graficaron mediante GraphViz.

Se capturaron redes de distintos tamaños utilizando distintas tecnologías a nivel enlance: Switched Ethernet y Wireless LAN


\subsection{Fuente S}
Esta fuente binaria está compuesta por los símbolos ${s_{Broadcast}, s_{Unicast}}$

\subsection{Fuente S1}

Esta fuente está compuesta por los símbolos $s_{Src}$, es decir, las direcciones IP origen de los paquetes ARP.